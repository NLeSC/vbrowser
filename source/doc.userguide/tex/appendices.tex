%
% VBrowser
%

\appendix
\chapter{Appendices}
\label{chap:appendices}

\section{VRL specification} 

This appendix specifies the syntax used for VRLs (Virtual Resource Locators)
which follows the specification of URIs (Universal Resource Indicators) as
specified in \cite{bernerslee2005uri}.\\
In short a VRL is an URI, but not all URIs are VRLs.\\
\par
VRLs can be seen as Virtual URLs which may have extra schemes which do not map
to an actual network protocol but to an 'Virtual' Protocol or 'VRS' Driver
protocol (VDriver). An example is 'rfts' which is not an actual network protocol but is
implemented as Grid Service protocol to an Reliable File Transfer Service. The
actual network protocol used is SOAP messages over https connections as
specified in the WSRF specifications. This is hidden for the user. \\

\subsection{Syntax} 

\hspace*{10mm}\begin{minipage}{170mm}
\begin{verbatim}
 VRL                ::= SCHEME + ':' + SCHEMESPECIFICPART

 SCHEMESPECIFICPART ::=   [ '//' + AUTHORITY ]  
                        + [ '/' + PATH ] 
                        + [ '?' + QUERY ] 
                        + [ '#' + FRAGMENT ]

 AUTHORITY          ::=  [ USERINFO + '@' ] + [ HOSTNAME + [ ':' + PORT ] ]  

 SCHEME             ::= 'file', 'gftp', 'sftp', 'http', 'lfn', 'srm', ..

 USERINFO           ::= USERNAME + [ '.' + DOMAINNAME ] + [ ':' + PASSWORD ]

 USERNAME           ::= UNRESERVED <see rfc3986> 

 DOMAINNAME         ::= UNRESERVED <see rfc3986>

 HOSTNAME           ::= UNRESERVED <see rfc3986> 

 PASSWORD           ::= UNRESERVED <see rfc3986>

 QUERY              ::= UNRESERVED <see rfc3986>
	
 FRAGMENT           ::= UNRESERVED <see rfc3986>

 UNRESERVED         ::= [ ALPHA | NUM |  '-' | '.' | '_' | '~' ] * 

\end{verbatim}
\end{minipage}\\
\\
\Note{Notes:}
\begin{itemize}
%   \item SRB URIs have a \Variable{DOMAINNAME} in the \Variable{USERNAME} part. 
%         VRLs will take the last dot separated string as domain name. 
%         For example in 'john.doe.vle', the 'vle' part is the domain name 
%         and 'john.doe' the username. 
  \item It is not recommended putting plain text passwords in URIs as this 
        is a security hazard. 
  \item The AUTHORITY part start with two slashes '//' but is optional. When
        the scheme specific part starts with only one slash it has no authority
        part.
\end{itemize}

\subsection{Examples of URIs}

This section explains how URIs (VRLs) are used in VLET/VBrowser .\\  
Below an example of a full URI.

\hspace*{10mm}\begin{minipage}{170mm}
\begin{verbatim}
     foo://john_doe@example.com:8042/over/there?name=ferret#nose
     \_/   \_______________________/\_________/ \_________/ \__/
      |              |                |            |         |
    scheme       authority           path        query    fragment
\end{verbatim}
 \end{minipage}\\

After the first colon (':') the \emph{scheme specific} part starts. 
This part can contain an authority part which always starts with two slashes
('//').\\
For example:\\
\\
  \tab \path{sftp://skoulouz@ui.grid.sara.nl:22/data/home/skoulouz}\\
\\
The \emph{authority} part can be omitted if for the scheme no authority is
needed. In that case the scheme specific part may NOT start with two slashes,
but only one in for example in the following URI:\\
\\
   \tab (1) \path{file:/home/dexter/e-laboratory}\\
\\
This is equivalent with an empty \emph{authority} URI as follows:\\
\\
   \tab (2) \path{file:///home/dexter/e-laboratory}\\ 
\\ 
The above URI (2) will be normalized to (1) to indicate an empty authority
part and remove superfluous slashes.\\
\\
\subsection{URI Attributes}

An URI can have URI attributes specified in the \emph{query} part. Some scheme
implementations support URI attributes to specify extra options or parameters
to the underlying implementation. \\
The syntax is as follows:\\ 
\\
\hspace*{10mm}\begin{minipage}{170mm}
\begin{verbatim}
 QUERY         ::= [ ParameterList ]  

 ParameterList ::=  Parameter + [ '&' + ParameterList ]

 Parameter     ::=   Name + '=' +  Value

\end{verbatim}
 \end{minipage}\\
\\
\\
Example of two URI Attributes in an LFC URI:\\
\\
\hspace*{10mm}\begin{minipage}{170mm}
\begin{verbatim}
 lfn://lfc.grid.sara.nl/grid/vlemed/stuff\
      ?lfc.listPreferredSEs=srm.grid.sara.nl\
      &lfc.replicaSelectionMode=Preferred
\end{verbatim}
\end{minipage}\\
\\
% \\
% Or specifying properties for LFC resources: \\
% \\
% \hspace*{10mm}\begin{minipage}{170mm}
% \begin{verbatim}
%  lfn://lfc.grid.sara.nl/grid/vlemed/stuff\
%      ?lfc.listPreferredSEs=srm.grid.sara.nl
% \end{verbatim}
% \end{minipage}\\
% \\
(Remove the backslash '\bsl' when concatenating the URI parts). \\
You can use this feature when invoking the \path{uricopy.sh} script or accessing
the VRS programmatically.
\par
Most properties which can be specified as Resource Properties using the
VBrowser can be specified here although not all combinations have been tested
and some are implementation specific. Check the documentation of the specific
protocol implementation to see which property is supported. 
\par
Don't forget to prepend the property with the name of the scheme, like in
\path{'lfc.listPreferredSEs'} to indicate the \path{lfc} property
\path{listPreferredSEs}. That is when lfc is replicating a file, it will use the storage elements indicated after the \path{'lfc.listPreferredSEs='} which for the example above would be srm.grid.sara.nl

%%%
%%%
%%%

\newpage
\section{Overview of configuration files and directories}

The next sections show an overview of the distribution and user
configuration files. See chapter \ref{chap:customization} for the
customization of these files.

\subsection{Installation files and settings}

 Selection of configuration files and directories in the VLET distribution
 replace \path{VLET_INSTALL} with the location of the distribution):\\
 \\
 \begin{tabular}{ l l }
   \path{VLET_INSTALL/etc} & Directory where all distribution settings are.\\
   \path{VLET_INSTALL/etc/vletrc.prop} & Installation properties and (default)
        settings.\\
   \path{VLET_INSTALL/etc/certificates} & Extra trusted CA root certificates.\\
   \path{VLET_INSTALL/etc/vomsdir/voms.xml} & VO Server configuration.\\
   \path{VLET_INSTALL/lib/plugins} & Custom VDriver plugins and VBrowser
        Viewers.\\
 \end{tabular}
 

\subsection{Default installation configuration file vletrc.prop} 

The default settings for SRB and for LFC can be found in
\path{VLET_INSTALL/etc/vletrc.prop}.\\
These are the default properties used when creating a new resource or contact
remote resources.\\
Most of these properties can also be specified during startup or as extra URI
attribute in the URI (VRL).\\
\\
% \subsubsection{Default SRB settings} 
% 
% Below an example of SRB properties found in
% \path{VLET_INSTALL/etc/vletrc.prop}. \\
% \\
% \begin{boxedlisting}
% \begin{verbatim}
% 
% ###
% # Default SRB host at SARA :
% #
% 
% srb.hostname=srb.grid.sara.nl
% srb.path=/VLENL/home
% srb.port=50000
% srb.mdasCollectionHome=/VLENL/home
% srb.mdasDomainHome=vlenl
% srb.mdasDomainName=vlenl
% srb.defaultResource=vleGridStore
% #srb.username=
% srb.mcatZone=VLENL
% # set to true when behind a firewall (Default)
% # if incoming connection are allowed set to 'false'
% srb.passiveMode=true
% #this string MUST match the Enum Value for 'GSI_AUTH'
% srb.AUTH_SCHEME=GSI_AUTH
% 
% \end{verbatim} 
% \end{boxedlisting}

\subsubsection{Default LFC settings} 

The default settings for LFC which can be found in
\path{VLET_INSTALL/etc/vletrc.prop}. \\
\\
\begin{boxedlisting}
\begin{verbatim}

## Default LFC Server settings: 
## 
# Default LFC hostname: Specify empty field for the user to fill in. 
#lfc.hostname=lfc.grid.sara.nl
lfc.hostname=<LFCHOST>
# Default port: 
lfc.port=5010
## Example of preconfigured Storage Elements:
#lfc.listPreferredSEs=srm.grid.sara.nl,tbn18.nikhef.nl
lfc.replicaNrOfTries=5
## Replica selection mode (reading). 
## One of: Preferred, PrefferedRandom, AllSequential, AllRandom
#lfc.replicaSelectionMode=PreferredRandom
## Replica creation mode (writing) 
## One of:  Preferred, PrefferedRandom, DefaultVO, DefaultVORandom
#lfc.replicaCreationMode=Preferred
## replica naming policy. One of: Similar,Random 
#lfc.replicaNamePolicy=Similar

\end{verbatim} 
\end{boxedlisting}

\subsection{User configuration files}
 
 All the user configurations are stored in the users home directory
 under \path{.vletrc}.\\
 An overview of current used directories and files are (replace HOME with the
 user's home directory):
 \\
 \\
 \begin{tabular}{ l l }
   \path{HOME/.vletrc/} & Directory where user configuration files are stored.\\
   \path{HOME/.vletrc/vletrc.prop} & User property settings.\\
   \path{HOME/.vletrc/cacerts} & User trusted certificates.\\
   \path{HOME/.vletrc/certificates/} & Directory for extra user trusted (CA) certificates.\\ 
   \path{HOME/.vletrc/guisettings.prop} & User GUI (VBrowser) settings.\\
   \path{HOME/.vletrc/mime.types} & User configured mime types.\\
   \path{HOME/.vletrc/viewerconf.prop} & User mime type to Viewer mapping.\\
   \path{HOME/.vletrc/myvle/} & User's 'My Grid' environment.\\
   \path{HOME/.vletrc/icons/} & User customized icons.\\
   \path{HOME/.vletrc/icons/mimetypes/} & User mimetype icons.\\
   \path{HOME/.vletrc/icons/mimetypes/text-plain.png} & Example User
      customized icon for mime type 'text/plain'.\\
   \path{HOME/.vletrc/plugins/} & User Installed plugins.\\

 \end{tabular}
\\

\subsection{Example user customized mime.types file}

The file \path{HOME/.vletrc/mime.types} contains 
user customized mime types. \\
See \ref{chap:customization}  how to customize this file.\\ 
\\
  \begin{boxedlisting}[170]
\begin{verbatim}
###
# File    : mime.types
# Location: $HOME/.vletrc/mime.types
# ---
# Default mime types:  <MIMETYPE>  <EXTENSION1> <EXTENSION2> ...
#

# example vle text mime type: 
application/vle-text          vletxt txt TXT

# example vlemed mime types:  
application/vlemed-fslview    nii nii.gz NII NII.GZ
application/jglite-jobids     vljids
application/taverna-scufl     scufl
application/feat-fsf          fsf FSF 

#end default mime types file 
\end{verbatim}
\end{boxedlisting}
 
\subsection{Example user customized viewerconf.prop file}

The file \path{HOME/.vletrc/viewerconf.prop} contains 
user customized mime type to (VBrowser) Viewer mappings.\\ 
See \ref{chap:customization}  how to customize this file. \\
\\
\begin{boxedlisting}
\begin{verbatim}
##
# File    : viewerconf.prop 
# Location: $HOME/.vletrc/viewerconf.prop
#---
# Mimetype/viewer class mapping file. NO spaces between '=' !.  
# <MIMETYPE>=<VIEWERCLASS> 

# Example viewer mapping for the application/vle-text mimetype:
application/vle-text=nl.uva.vlet.gui.viewers.TextViewer

# Example vlemed mimetype and to viewer mapping: 
application/vlemed-fslview=nl.amc.vlet.feat.fslview.gui.ViewerFSL
application/taverna-scufl=nl.uva.vlet.moteur.plugin.gui
application/jglite-jobids=nl.uva.vlet.monitoring.gui.ViewerGliteJobMonitoring
application/feat-fsf=nl.amc.vlet.feat.client.main.gui.ViewerFeatParameterSweep

# end viewerconf.prop file 
\end{verbatim}
\end{boxedlisting}\\
\\

 
 